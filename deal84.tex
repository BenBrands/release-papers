\documentclass{ansarticle-preprint}
%\usepackage{ucs}
\usepackage[utf8]{inputenc}
\usepackage{amsmath}
%\usepackage{cite}
\usepackage{anslistings}

\usepackage{fontenc}
\usepackage{graphicx}

\hypersetup{
  pdfauthor={Wolfgang Bangerth, Timo Heister, Luca Heltai, Guido Kanschat,
    Martin Kronbichler, Matthias Maier, Bruno Turcksin
  %, Toby D. Young
  },
  pdftitle={The deal.II Library, Version 8.4.0, 2016},
}

\newcommand{\specialword}[1]{\texttt{#1}}
\newcommand{\dealii}{{\specialword{deal.II}}}
\newcommand{\pfrst}{{\specialword{p4est}}}
\newcommand{\trilinos}{{\specialword{Trilinos}}}
\newcommand{\aspect}{\specialword{Aspect}}
\newcommand{\petsc}{\specialword{PETSc}}
\newcommand{\cmake}{{\specialword{CMake}}}
\newcommand{\autoconf}{{\specialword{autoconf}}}

\title{The \dealii{} Library, Version 8.4.0}
\author[1]{Wolfgang Bangerth}
\affil[1]{Department of Mathematics, Texas A\&M University, College Station,
    TX 77843, USA,
    {\texttt{bangerth@math.tamu.edu}}}
\author[2]{Denis Davydov}
\affil[2]{Chair of Applied Mechanics, University of
    Erlangen-Nuremberg, Egerlandstr. 5, 91058 Erlangen, Germany,
  {\texttt{denis.davydov@fau.de}}}
\author[3]{Timo Heister}
\affil[3]{Mathematical Sciences,
  O-110 Martin Hall.
  Clemson University.
  Clemson, SC 29634, USA,
  {\texttt{heister@clemson.edu}}}
\author[4]{Luca Heltai}
\affil[4]{SISSA - International School for Advanced Studies, Via
  Bonomea 265, 34136 Trieste, Italy,
  {\texttt{luca.heltai@sissa.it}}}
\author[5]{Guido Kanschat}
\affil[5]{Interdisciplinary Center for Scientific Computing (IWR),
  Universit{\"a}t Heidelberg, Im Neuenheimer Feld 368, 69120 Heidelberg, Germany,
  {\texttt{kanschat@uni-heidelberg.de}}}
\author[6]{Martin Kronbichler}
\affil[6]{Institute for Computational Mechanics, Technische
  Universit{\"a}t M{\"u}nchen, Boltzmannstr.~15, 85748 Garching b. Mænchen,
  Germany,
  {\texttt{kronbichler@lnm.mw.tum.de}}}
\author[7]{Matthias Maier}
\affil[7]{School of Mathematics, University of Minnesota, 127 Vincent Hall,
  206 Church Street SE, Minneapolis, MN 55455, USA,
  {\texttt{msmaier@umn.edu}}}
\author[8]{Bruno Turcksin}
\affil[8]{Department of Mathematics, Texas A\&M University, College Station,
  TX 77843, USA,
  {\texttt{turcksin@math.tamu.edu}}}
\author[9]{David Wells}
\affil[9]{Department of Mathematical Sciences, Rensselaer Polytechnic Institute,
Troy, NY 12180, USA
  {\texttt{wellsd2@rpi.edu}}}
%\author[8]{Toby~D.~Young}
%\affil[8]{Institute of Fundamental Technological Research of the Polish
%  Academy of Sciences, ul. Pawi{\'n}skiego 5b, Warsaw 02-106, Poland,
%  {\texttt{tyoung@ippt.pan.pl}}}

\renewcommand{\labelitemi}{--}


\begin{document}
\maketitle

\begin{abstract}
  This paper provides an overview of the new features of the finite element
  library \dealii{} version 8.4.0.
\end{abstract}


\section{Overview}

\dealii{} version 8.4.0 was released February 29, 2016. This paper provides an
overview of the new features of this release and serves as a citable
reference for the \dealii{} software library version 8.4.0. \dealii{} is an
object-oriented finite element library used around the world in the
development of finite element solvers. It is available for free under the
GNU Lesser General Public License (LGPL) from the \dealii{} homepage at
\url{http://www.dealii.org/}.

The major changes of this release are:
\begin{itemize}
\item Parallel triangulations can now be partitioned in ways that allow
  weighting cells differently.
\item Improved support for mixed-type arithmetic throughout the library.
\item A new triangulation type that supports parallel computations
  but ensures that the entire mesh is available on every processor.
\item An implementation of the Rannacher-Turek element, as well as an
  element that extends the usual Q(p) elements by bubble functions.
\item Second and third derivatives of finite element fields are now
  computed exactly.
\item The various \emph{Concepts}, or requirements on template parameters
  in the library, are now consistently labeled and documented as such.
\item The interface between finite elements, quadrature, mapping, and the
  FEValues class has been rewritten. It is now much better documented.
\item Initial support for compiling with Visual C++ 2013 and 2015 under Windows
  has been added.
\item  More than 140 other features and bugfixes.
\end{itemize}
Some of these will be detailed in the following section.
Information on how to cite \dealii{} is provided in Section \ref{sec:cite}.


\section{Significant changes to the library}

This release of \dealii{} contains a number of large and significant changes
that will be discussed in the following sections. It of course also contains a
vast number of smaller changes and added functionality; the details of these
can be found
\href{https://www.dealii.org/8.4.0/doxygen/deal.II/changes_between_8_3_0_and_8_4.html}{in the file that lists all changes for this release}
and that is linked to from the web site of each release as well as the
release announcement.


\subsection{Parallel triangulations can now be partitioned with weights}

Previously, partitioning a parallel mesh (represented by the
\texttt{parallel::distributed::Triangulation} class) between
processors assumed that every cell should be weighted equally. On the
other hand, the \pfrst{} library which manages the partitioning
process, allows for attaching weights for each cell and thereby
enables ways in which not the number of cells per MPI process is
equilibrated, but the sum of weights on the cells managed by each
process. \dealii{} now also supports this feature.

The implementation of this mechanism is based on a callback mechanism,
in the form of the signal-slot design pattern. User codes can register
functions that will be called upon mesh refinement and coarsening,
returning a weight for each cell. These weights will be added after
all slots (i.e., callback functions) connected to the signal have been
called.

The mechanism chosen has the advantage that all parties that use a
triangulation, for example multiple \texttt{DoFHandler} objects or a
scheme that tracks particles that are advected along with a flow field
and stored them per-cell, can indicate their computational needs for
each cell, and there no central place in a user code (other than the
triangulation itself) that has to collect these needs and forward this
information.


\subsection{Improved support for mixed-type arithmetic throughout the
  library}

When evaluating finite element fields or their derivatives at
a points $\mathbf x_q$, one typically has to do an operation of the form
\begin{align*}
  u_h(\mathbf x_q) &= \sum_{j} U_j \varphi_j(\mathbf x_q),
  \\
  \nabla u_h(\mathbf x_q) &= \sum_{j} U_j \nabla\varphi_j(\mathbf x_q).
\end{align*}
The value or derivatives of shape functions, $\varphi_j(\mathbf x_q)$
or $\nabla\varphi_j(\mathbf x_q)$ are internally evaluated by classes
derived from the \texttt{FiniteElement} and \texttt{Mapping}, and are
computed as scalars or tensors of type \texttt{double}. On the other
hand, the expansion coefficients of the field, $U_j$ are stored in
vectors of types chosen by the user; their underlying representation
can be \texttt{double}, but also \texttt{float}, \texttt{long double},
\texttt{PetscScalar}, or \texttt{std::complex<float>} for example. It could also
be an autodifferentiation type.

To facilitate the correct typing of the computed quantity, \dealii{}
now contains mechanisms by which one can evaluate the \textit{type} of
the product of two values, and this type is now consistently used
throughout the library in expressions such as those above. Thus, as an
example, if the user stores the expansion coefficients in a
\texttt{Vector<long double>}, then the gradients
$\nabla\varphi_j(\mathbf x_q)$ will be computed as objects of type
\texttt{Tensor<1,dim,long double>}. Likewise, if a user uses a PETSc
vector, and PETSc was configured with complex scalar types, then the
second derivatives of the solution field will be computed as
\texttt{SymmetricTensor<2,dim,PetscScalar>}, which should equal
\texttt{SymmetricTensor<2,dim,std::complex<double> >}.

These improvements make type-correct computations possible in many
places. In particular, this enables the use of complex-valued solution
vectors in many more places. On the other hand, many but not all
places have learned what actually to do with complex numbers. This is,
in particular, true for the \texttt{DataOut} class that generates an
intermediate data format that can then be written to graphical output
files for visualization. Since no format we are aware of supports
complex numbers, future versions still need to learn how to separate
real and imaginary parts of complex numbers, and output them as two
components.


\subsection{A new ``shared'' triangulation type for parallel computations}

Previously, in order to use \texttt{Triangulation<dim,spacedim>} class with MPI one had to manually
partition triangulation (e.g. via \texttt{METIS}), calculate and store index sets of locally owned and locally relevant degrees-of-freedom (DoF), querry \texttt{subdomain\_id}  of a cell during assembly etc.
In order to facilitate the usage of \texttt{Triangulation} class with MPI  
and to make its behaviour in this context consistent with \texttt{parallel::distributed::Triangulation}, a new class
\texttt{parallel::shared::Triangulation} was introduced. 
It extends \texttt{Triangulation} class to automatically partition triangulation when run with MPI.
Identical functionality between \texttt{shared} and \texttt{distributed} triangulation  
(e.g. locally owned and relevant DoFs, mpi communicator, etc)
is grouped in the parent class \texttt{parallel::Triangulation}.
The main difference between the two parallel triangulations is that in the case of 
\texttt{parallel::shared::Triangulation} each process stores all cells of the triangulation.
Consequently, by default there are no artificial cells. 
That is, cells which are attributed to the current processor are marked as locally owned 
(\texttt{cell->is\_locally\_owned()} returns \texttt{true}) 
and the rest are ghost cells.
This behaviour can be altered via an additional boolean flag provided to the constructor of the class. 
In this case, a set of ghost cells will consist of a halo layer of cells around locally owned cells.
Cells which are neither ghost nor locally owned are marked as artificial.
This is consistent with the behaviour of \texttt{parallel::distributed::Triangulation},
although in the latter case the size of the set of artificial cells will be much smaller.
The introduction of \texttt{parallel::shared::Triangulation} class together with the 
optional artificial cells and parent \texttt{parallel::Triangulation} class
facilitates writing
algorithms that are indifferent to the way triangulation is stored in the MPI context. 
For example, \texttt{DoFTools::locally\_relevant\_dofs()} will now return a subset of 
all DoFs for both triangulations, granted that artificial cells are enabled in 
\texttt{parallel::shared::Triangulation}.
Assmebly routines can now use \texttt{cell->is\_locally\_owned()} 
for both triangulations.
Currently, only standard (non hp) \texttt{DoFHandler} is supported.

\subsection{Second and third derivatives of finite element fields are now
  computed exactly}

Second derivatives of solution fields, i.e., 
\begin{align*}
  \nabla^2 u_h(\mathbf x_q) &= \sum_{j} U_j \nabla^2\varphi_j(\mathbf x_q)
\end{align*}
were previously computed by finite differencing of first
derivatives. The reason for this approach is that in order to compute
the second derivatives of shape functions, $\nabla^2\varphi_j(\mathbf
x_q)$ one needs (at least) derivatives of the inverse of the Jacobian of the
mapping. This is easy to see because, for the usual $Q_p$ Lagrange
elements, one has that $\nabla\varphi_j(\mathbf
x_q)=J^{-1}\hat\nabla\hat\varphi_j(\hat{\mathbf x}_q)$, where quantities
with a hat refer to coordinates and functions on the reference cell,
and $J$ is the Jacobian of the mapping from reference to real
cell. Thus, the second derivatives satisfy
\begin{align*}
  \nabla^2\varphi_j(\mathbf x_q)
  =
  J^{-1}\hat\nabla\left[J^{-1}\hat\nabla\hat\varphi_j(\hat{\mathbf x}_q)\right]
  =
  J^{-1} J^{-1}\hat\nabla^2\hat\varphi_j(\hat{\mathbf x}_q)
  +
  J^{-1}\left(\hat\nabla J^{-1}\right)\hat\nabla\hat\varphi_j(\hat{\mathbf x}_q),
\end{align*}
where in the last expression, matrices and tensors of rank 1 and 3
have to be appropriately contracted.  Here, the difficulty lies in
computing the derivative $\hat\nabla J^{-1}$: for the usual mappings
on quadrilaterals and hexahedra, $J$ is in general a polynomial in the
reference coordinates $\hat {\mathbf x}$, so $J^{-1}$ is a rational
function. It is possible to compute the derivatives of this object,
but they are difficult and cumbersome to evaluate, especially for
higher order mappings.

The key to computing second (and higher) derivatives of shape
functions is to recognize that $\hat\nabla J^{-1}$ can be expressed
more conveniently by observing that
\begin{align*}
  0 &= \hat\nabla I
  \\
  &= \hat\nabla (JJ^{-1})
  \\
  &= J (\hat\nabla J^{-1}) + (\hat\nabla J) J^{-1},
\end{align*}
and consequently $\hat\nabla J^{-1} = - J^{-1} (\hat\nabla J)
J^{-1}$. Here, $J^{-1}$ is a matrix that is already available from
computing first derivatives, and $\hat\nabla J$ is an easily computed
rank-3 tensor with polynomial entries. Using this approach, we have
\begin{align*}
  \nabla^2\varphi_j(\mathbf x_q)
  =
  J^{-1} J^{-1}\hat\nabla^2\hat\varphi_j(\hat{\mathbf x}_q)
  -
  J^{-1} J^{-1} (\hat\nabla J)
  J^{-1}\hat\nabla\hat\varphi_j(\hat{\mathbf x}_q),
\end{align*}
again with an appropriate set of contractions over the indices of the
objects on the right.

This, and corresponding extensions to compute third derivatives, have
now been implemented in several of the finite element and mapping
classes by Maien Hamed, and are available through the
\texttt{FEValues} interface to shape functions and their derivatives.

\subsection{Visual C++ support}

The library can now be compiled under Windows with Visual C++ 2013 and
2015. The support is still experimental for the following reasons: First, we
currently only support static linking. This will slow down linking of
application code immensly.  Second, only a minimal testsuite is working, which
is mainly because static linking of thousands of test executables is not
viable. Therefore, we can not exclude the possibility of subtle bugs in the
library.  Finally, there is of course limited support for external packages.

\subsection{Incompatible changes}

\subsubsection{Revision of the interface between finite elements,
  quadratures, and mappings}

Finite element classes describe shape functions as continuous (as
opposed to discrete) objects on the reference cell. On the other hand,
in actual practice, one only needs information about shape functions
at finitely many quadrature points, and these are typically the same
on every cell in a loop over all cells. Furthermore, the evaluation of
shape functions at quadrature points then also needs to be mapped to
the real cell, typically using a polynomial mapping.

To facilitate this complex interplay, \dealii{} has three class
hierarchies with base classes \texttt{FiniteElement} (for the
description of shape functions on the reference cell),
\texttt{Quadrature} (for the locations and weights of quadrature
points on the reference cell), and \texttt{Mapping} (for the
description of mappings from the reference to the real cell). In
almost all cases, users create an object of a derived class for each
of these categories, but they never access any of the members of these
classes and instead leave this task to the \texttt{FEValues} class
(and \texttt{FEFaceValues}, \texttt{FESubfaceValues}) that presents
all one typically needs: the mapped values, gradients, and
higher order derivatives of shape functions at the quadrature points
of the real cell.

The interplay between these classes in the \texttt{FEValues} interface
is one of the oldest parts of the library. It was, at the time, not
designed based on specifications we explicitly or implicitly knew this
class had to satisfy, but instead organically grew to its current
state. These interfaces have now been fundamentally rewritten: some
functions have been replaced; several others had their argument lists
shuffled, sorted, and made more uniform; and everything has generally
been far better documented.

None of the changes in this arena is visible to the average user. The
only user codes that are affected in an incompatible way are those
that implement finite element or mapping classes.

\subsubsection{Other incompatible changes}

The
\href{https://www.dealii.org/8.4.0/doxygen/deal.II/changes_between_8_3_0_and_8_4.html}{file
  that lists all changes for this release} lists another 18
incompatible changes, but none of these should in fact be visible in
typical user codes. Some remove previously deprecated classes and
functions, and the majority change internal interfaces that are not
typically used in user codes.



\section{How to cite \dealii{}}\label{sec:cite}

In order to justify the work the developers of \dealii{} put into this
software, we ask that papers using the library reference one of the
\dealii{} papers. This helps us justify the effort we put into it.

There are various ways to reference \dealii{}. To acknowledge the use of a
particular version of the library, reference the present document. For up
to date information and bibtex snippets for this document see:

\bigskip

\begin{center}
 \url{https://www.dealii.org/publications.html}
\end{center}

\bigskip

% \begin{minipage}{0.9\textwidth}%no page break in here please
% \begin{verbatim}
% @article{dealII82,
%   title = {The {\tt deal.{I}{I}} Library, Version 8.2},
%   author = {W. Bangerth and T. Heister and L. Heltai
%    and G. Kanschat and M. Kronbichler and M. Maier
%    and B. Turcksin and T. D. Young},
%   journal = {preprint},
%   year = {2015},
% }
% \end{verbatim} %  journal = {arXiv preprint \url{http://arxiv.org/abs/TODO}},
% \end{minipage}

The original \texttt{\dealii{}} paper containing an overview of its
architecture is \cite{BangerthHartmannKanschat2007}. If you rely on specific
features of the library, please consider citing any of the following:
\begin{itemize}
 \item For geometric multigrid: \cite{Kanschat2004,JanssenKanschat2011};
 \item For distributed parallel computing: \cite{BangerthBursteddeHeisterKronbichler11};
 \item For $hp$ adaptivity: \cite{BangerthKayserHerold2007};
 \item For matrix-free and fast assembly techniques:
   \cite{KronbichlerKormann2012};
 \item For computations on lower-dimensional manifolds:
   \cite{DeSimoneHeltaiManigrasso2009};
 \item For integration with CAD files and tools:
   \cite{HeltaiMola2015};
 \item For \texttt{LinearOperator} and \texttt{PackagedOperation} facilities:
   \cite{MaierBardelloniHeltai-2015-a}.
 \item For uses of the \texttt{WorkStream} interface:
   \cite{TKB16}.
\end{itemize}

\dealii{} can interface with many other libraries:
\begin{itemize}
\item ARPACK \cite{arpack}
\item BLAS, LAPACK
\item HDF5 \cite{hdf5}
\item METIS \cite{karypis1998fast}
\item MUMPS \cite{ADE00,MUMPS:1,MUMPS:2,mumps-web-page}
\item muparser \cite{muparser-web-page}
\item NetCDF \cite{rew1990netcdf}
\item OpenCASCADE \cite{opencascade-web-page}
\item p4est \cite{p4est}
\item PETSc \cite{petsc-user-ref,petsc-web-page}
\item SLEPc \cite{Hernandez:2005:SSF}
\item Threading Building Blocks \cite{Rei07}
\item Trilinos \cite{trilinos,trilinos-web-page}
\item UMFPACK \cite{umfpack}
\end{itemize}
Please consider citing the appropriate references if you use interfaces to these
libraries.

Older releases of \dealii{} can be cited as \cite{dealII80,dealII81,dealII82,dealII83}.

\nocite{BangerthKanschat1999}

\section{Acknowledgments}

\dealii{} is a world-wide project with dozens of contributors around the
globe. Other than the authors of this paper, the following people contributed code to
this release:
%
% get this from the changes.h file using
%   egrep '^ *\(.*201[56789]' 8.3.0-vs-8.4.0.h | perl -p -e 's/201\d.*//g; s/, */\n/g; s/^ *\(?//g;' | sort | uniq
% then sort by last name and remove the authors of this paper
%
Daniel Arndt,
Mauro Bardelloni,
Alistair Bentley,
Andrea Bonito,
Claire Bruna-Rosso,
Krzysztof Bzowski,
Praveen Chandrashekar,
Conrad Clevenger,
Patrick Esser,
Rene Gassmoeller,
Arezou Ghesmati,
Maien Hamed,
Alexander Grayver,
Lukas Korous,
Aslan Kosakian,
Adam Kosik,
Konstantin Ladutenko,
Jean-Paul Pelteret,
Lei Qiao,
Gennadiy Rishin,
Angel Rodriguez,
Alberto Sartori,
Daniel Shapero,
Jason Sheldon,
Jan Stebel,
Florian Sonner,
Zhen Tao,
Heikki Virtanen,
Daniel Weygand


Their contributions are much appreciated!


\dealii{} and its developers are financially supported through a
variety of funding sources. W.~Bangerth and B.~Turcksin were partially
supported by the National Science Foundation under award OCI-1148116
as part of the Software Infrastructure for Sustained Innovation (SI2)
program; and by the Computational Infrastructure in Geodynamics initiative
(CIG), through the National Science Foundation under Award
No.~EAR-0949446 and The University of California -- Davis.

L.~Heltai was partially supported by the project OpenViewSHIP,
``Sviluppo di un ecosistema computazionale per la progettazione
idrodinamica del sistema elica-carena'', financed by Regione FVG - PAR
FSC 2007-2013, Fondo per lo Sviluppo e la Coesione, and by the project
TRIM ``Tecnologia e Ricerca Industriale per la Mobilit\`a Marina'',
CTN01-00176-163601, funded by MIUR -  Ministero dell'Istruzione,
dell'Universit\`a e della Ricerca.

T.~Heister was partially supported by the Computational Infrastructure in
Geodynamics initiative (CIG), through the National Science Foundation under Award No. EAR-0949446 and The University of California—Davis, and National Science Foundation grant DMS1522191.

The Interdisciplinary Center for Scientific Computing (IWR) at Heidelberg University has provided
hosting services for the \dealii{} web page and the SVN archive.

%\marginpar{Everyone's funding, please}


\bibliography{deal84}{}
\bibliographystyle{abbrv}

\end{document}
