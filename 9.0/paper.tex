\documentclass{ansarticle-preprint}
%\usepackage{ucs}
\usepackage[utf8]{inputenc}
\usepackage{amsmath}
%\usepackage{cite}
\usepackage{anslistings}
\usepackage{multicol}

\usepackage{pgfplots}
\usepackage{pgfplotstable}

\usepackage{fontenc}
\usepackage{graphicx}

\pgfplotsset{compat=1.9}

\newcommand{\specialword}[1]{\texttt{#1}}
\newcommand{\dealii}{{\specialword{deal.II}}}
\newcommand{\pfrst}{{\specialword{p4est}}}
\newcommand{\trilinos}{{\specialword{Trilinos}}}
\newcommand{\aspect}{\specialword{Aspect}}
\newcommand{\petsc}{\specialword{PETSc}}
\newcommand{\cmake}{{\specialword{CMake}}}
\newcommand{\autoconf}{{\specialword{autoconf}}}

%
% Author list -- please add yourself in both places below (in
%                alphabetical order) if you think that your
%                contributions to the last release warrant this
%

\hypersetup{
  pdfauthor={
    Giovanni Alzetta,
    Daniel Arndt,
    Vishal Boddu,
    Benjamin Brands,
    Denis Davydov,
    Rene Gassm\"{o}ller,
    Timo Heister,
    Luca Heltai,
    Martin Kronbichler
    Matthias Maier,
    David Wells
  },
  pdftitle={The deal.II Library, Version 9.0, 2018},
}

\title{The \dealii{} Library, Version 9.0}

\author[1]{Giovanni Alzetta}
\affil[1]{SISSA,
  International School for Advanced Studies,
  Via Bonomea 265,
  34136, Trieste, Italy.
{\texttt{galzetta@sissa.it}}}

\author[2]{Daniel Arndt}
\affil[2]{Interdisciplinary Center for Scientific Computing,
  Heidelberg University,
  Im Neuenheimer Feld 205,
  69120 Heidelberg, Germany.
  {\texttt{daniel.arndt@iwr.uni-heidelberg.de}}}

\author[3]{Wolfgang Bangerth}
\affil[3]{Department of Mathematics, Colorado State University, Fort
  Collins, CO 80523-1874, USA.
    {\texttt{bangerth@colostate.edu}}}

\author[4]{Vishal Boddu}
\affil[4]{Chair of Applied Mechanics,
  Friedrich-Alexander-Universit\"{a}t Erlangen-N\"{u}rnberg,
  Egerlandstr.\ 5,
  91058 Erlangen, Germany.
  {\texttt{vishal.boddu@fau.de}}}
  
\author[5]{Benjamin Brands}
\affil[5]{Chair of Applied Mechanics,
          Friedrich-Alexander-Universit\"{a}t Erlangen-N\"{u}rnberg,
          Egerlandstr.\ 5, 91058 Erlangen, Germany.
          {\texttt{benjamin.brands@fau.de}}}

\author[6]{Denis Davydov}
\affil[6]{Chair of Applied Mechanics, Friedrich-Alexander-Universit\"{a}t Erlangen-N\"{u}rnberg,
            Egerlandstr.\ 5, 91058 Erlangen, Germany.
    {\texttt{denis.davydov@fau.de}}}

\author[7]{Rene Gassm\"{o}ller}
\affil[7]{Department of Earth and Planetary Sciences,
  University of California Davis,
  One Shields Avenue,
  CA-95616 Davis, USA.
  {\texttt{rgassmoeller@ucdavis.edu}}}

\author[8]{Timo Heister}
\affil[8]{Mathematical Sciences,
  O-110 Martin Hall,
  Clemson University,
  Clemson, SC 29634, USA.
  {\texttt{heister@clemson.edu}}}
%
\author[9]{Luca Heltai}
\affil[9]{SISSA,
  International School for Advanced Studies,
  Via Bonomea 265,
  34136, Trieste, Italy.
{\texttt{luca.heltai@sissa.it}}}

\author[10]{Martin Kronbichler}
\affil[10]{Institute for Computational Mechanics,
  Technical University of Munich,
  Boltzmannstr.~15, 85748 Garching, Germany.
  {\texttt{kronbichler@lnm.mw.tum.de}}}

\author[11]{Matthias Maier}
\affil[11]{School of Mathematics,
  University of Minnesota,
  127 Vincent Hall, 206 Church Street SE,
  Minneapolis, MN 55455, USA.
  {\texttt{msmaier@umn.edu}}}

\author[12]{Jean-Paul Pelteret}
\affil[12]{Chair of Applied Mechanics,
  Friedrich-Alexander-Universit\"{a}t Erlangen-N\"{u}rnberg,
  Egerlandstr.\ 5,
  91058 Erlangen,
  Germany.
  {\texttt{jean-paul.pelteret@fau.de}}}

 \author[13]{Bruno Turcksin\footnote{
   This manuscript has been authored by UT-Battelle, LLC under Contract No.
   DE-AC05-00OR22725 with the U.S. Department of Energy. The United States
   Government retains and the publisher, by accepting the article for
   publication, acknowledges that the United States Government retains a
   non-exclusive, paid-up, irrevocable, worldwide license to publish or reproduce
   the published form of this manuscript, or allow others to do so, for United
   States Government purposes. The Department of Energy will provide public
   access to these results of federally sponsored research in accordance with the
   DOE Public Access Plan (http://energy.gov/downloads/doe-public-access-plan).}}
 \affil[13]{Computational Engineering and Energy Sciences Group,
   Computional Sciences and Engineering Division,
   Oak Ridge National Laboratory, 1 Bethel Valley Rd.,
   TN 37831, USA.
   {\texttt{turcksinbr@ornl.gov}}}

\author[14]{David Wells}
\affil[14]{Department of Mathematical Sciences, Rensselaer Polytechnic
Institute, Troy, NY 12180, USA.
  {\texttt{wellsd2@rpi.edu}}}

\renewcommand{\labelitemi}{--}




\begin{document}
\maketitle

\begin{abstract}
  This paper provides an overview of the new features of the finite element
  library \dealii{} version 9.0.
\end{abstract}


\section{Overview}

\dealii{} version 9.0.0 was released April X, 2018. This paper provides an
overview of the new features of this major release and serves as a citable
reference for the \dealii{} software library version 9.0. \dealii{} is an
object-oriented finite element library used around the world in the
development of finite element solvers. It is available for free under the
GNU Lesser General Public License (LGPL) from the \dealii{} homepage at
\url{http://www.dealii.org/}.

The major changes of this release are:
\begin{itemize}
\item
  Manifold smoothing: The manifold smoothing algorithms applied in the
  Triangulation class and MappingQGeneric have been changed from the old
  Laplace-style smoothing to a transfinite interpolation that linearly
  blends between the descriptions on the faces around a cell. The old
  transformation introduced boundary layers inside cells that prevented
  convergence rates from exceeding \(3.5\) in the global \(L^2\) errors on typical
  settings. This change also considerably improves mesh quality on settings
  where curved descriptions are only applied to the boundary rather than
  the whole volume.

\item
  \dealii{} first offered support for a subset of C++11 features in
  version 6.2, released in 2009. The current release is the first to \emph{require} a compiler
  supporting C++11. \dealii{} now uses language improvements such as using
  \texttt{emplace\_back()} instead of \texttt{push\_back()}, moving objects
  instead of copying them, using \texttt{nullptr} instead of \texttt{NULL}, and
  marking members with \texttt{=delete;} and \texttt{=default;} where
  appropriate. These changes include some minor incompatibilities: all
  \texttt{clone()} functions (such as \texttt{FiniteElement::clone()} and
  \texttt{Mapping::clone()}) now return \texttt{std::unique\_ptr}s instead of
  C-style raw pointers. Indeed, nearly every interface that returns a pointer
  now returns either a \texttt{std::shared\_ptr} or \texttt{std::unique\_ptr},
  which clarifies object ownership responsibilities and avoids memory leaks.

\item \dealii{} has made extensive use of both the clang-tidy and coverity
      static analysis tools for detecting bugs and other issues in the code.
      %% TODO is there somewhere we can cite for either of these tools? It would
      %% be nice to recognize the department of homeland security for offering
      %% coverity.
      %% TODO at the time of writing this coverity is down; get an issue count
      %% when it is back online.

\item
  \texttt{LinearOperator}, a flexible template class that implements the action of a
  linear operator (see \cite{MaierBardelloniHeltai-2016-b}), now supports
  computations with Trilinos, Schur complements, and linear constraints. This
  class is, as of this release, the official replacement for about half a dozen
  similar (but less general) classes, such as \texttt{FilteredMatrix},
  \texttt{IterativeInverse}, and \texttt{PointerMatrix}.

\item
  Significant extension of matrix-free capabilities, including support for face
  integrals in discontinuous Galerkin schemes, a new Hermite-like polynomial
  basis adapted to face integrals that involve derivatives of shape functions,
  and several performance enhancements.

\item
  Support for GPU

\item
  In this version we add template wrappers for ScaLAPACK -- parallel dense linear algebra
  library with block-cyclic matrix distribution over 2D process grid.
  The functionality and interface is similar to the LAPACK wrappers,
  matrix-matrix multiplication, Cholesky and LU factorizations,
  eigensolvers, SVD, pseudoinverse, save/load in HDF5 format and
  other functionality is implemented.
  A user can combine the new linear algebra with other MPI-parallel
  algorithms in \dealii{} without the need to worry about MPI communicator
  being divisible exactly into 2D process grid. The processes which are not part of
  the grid are labeled inactive and ScaLAPACK routines will not be called on them.
  %
  Additionally, we improved support of LAPACK. We added methods to perform
  rank-1 update/downdate, calculate Cholesky factorization, trace and determinant,
  as well as estimate the reciprocal condition number.
  We also now support configuration with 64-bit BLAS.

\item
  A dedicated sundials module has been created to provide support for the
  SUNDIALS library. SUNDIALS is a SUite of Nonlinear and
  DIfferential/ALgebraic equations Solvers. ARKode, IDA, and KINSOL are
  supported. ARKode is a solver library that provides adaptive-step time
  integration. IDA is a package for the solution of differential-algebraic
  equations systems in the form $F(t,y,y')=0$. KINSOL is solver for nonlinear
  algebraic systems.

\item
  A dedicated optimization/rol module has been created to provide support for
  the Rapid Optimization Library (ROL). ROL is a package for large-scale
  optimization.

\item
  A dedicated particles module has been created to provide support for the
  storage of particles and their properties. This module provides a base class
  \texttt{Particles} which represents a particle with position, an ID number and
  a variable number of properties. It also provides a \texttt{ParticleHandler}
  class that manages the storage and handling of particles. This class
  distributes the particles in distributed domains (these are the only domains
  supported by this class so far)

\item
  A dedicated differentiation module has been created to provide support for
  automatic and symbolic differentiation. Automatic differentiation is a set of
  technique to evaluate the derivatives of a function defined by a compute
  program. Currently the Adol-C and Sacado libraries are supported through a
  unified interface. In practice, this support means that Adol-c and Sacado
  data types can be used in the \texttt{FEValues} and \texttt{FEValuesViews}
  to represent degree-of-freedom values.

\item
  Support for complex-valued vectors is now at the same level as real-valued
  vectors.

\item
  Support for the nanoflann library. nanoflann is a library for building
  KD-trees of datasets and for the fast querying of closest neighbors in
  KD-trees. nanoflann is used to speed up the search of the p nearest neighbors
  of a given point, or searching the points that fall within a radius of a
  target point.

\item
  Support for the Open Asset Import Library (Assimp). Assimp can be used to read
  about 40 different 3D graphics formats. A subset of these formats can be used
  to generate two-dimensional meshes possibly embedded in a three-dimensional
  space.

\item
  Initial support for Gmsh

\item
  Initial support for generalized support points: Internal interpolation
  rewritten (support for $H(\text{curl})$ and $H(\text{div})$ conforming
  elements).

\item New python tutorial program tutorial-1; as well as
  updates to step-37. In addition, the separate code
  gallery of \dealii{} has gained a number of new entries.

  \item More than 330 other features and bugfixes.
\end{itemize}
The more important ones of these changes will be detailed in the
following section.  Information on how to cite \dealii{} is provided
in Section \ref{sec:cite}.



\section{Significant changes to the library}

This release of \dealii{} contains a number of large and significant changes
that will be discussed in the following sections. It of course also contains a
vast number of smaller changes and added functionality; the details of these
can be found
\href{https://www.dealii.org/developer/doxygen/deal.II/changes_between_8_5_and_9_0.html}{
in the file that lists all changes for this release}, see \cite{changes90}.
(The file is also linked to from the web site of each release as well as
the release announcement.)


\subsection{Feature 1}

\subsection{Feature 2}


\subsection{New and updated tutorial programs}
\marginpar{OLD}
In addition to the updated tutorial programs mentioned in the previous
section, this release of \dealii{} includes three new tutorials TODO: old:
\begin{itemize}
 \item {\bf step-55} explains how to solve the Stokes
 equations efficiently in parallel. It is a good introduction to solving
 systems of PDEs in parallel, discusses optimal block
 preconditioners, and demonstrates other aspects like error computation.
 Inverses of individual blocks of the linear system are approximated with an algebraic
 multigrid preconditioner.

 \item {\bf step-56} shows how to apply geometric multigrid
 preconditioners on a subset of a system of PDEs. The problem solved here is the
 Stokes equations, like in step-55.

 \item {\bf step-57} solves the stationary Navier-Stokes equations.
 The nonlinear system is solved using Newton's method on a sequence of adaptively refined
 grids. The preconditioner is again built on a block factorization of the saddle point
 system like in step-55 and step-56, but the non-symmetric terms stemming from the
 nonlinear convective part requires more sophisticated solvers. The benchmark problem,
 flow in the 2d lid-driven cavity, requires a continuation method for high Reynolds
 numbers.
 \end{itemize}

In addition to tutorials, \dealii{} has a separate ``code gallery'' that
consists of programs shared by users as examples of what can be
done with \dealii{}. While not part of the release process, it is nonetheless
worth mentioning that the set of new programs since the last release covers
the following topics:
\marginpar{OLD}
  \begin{itemize}
    \item Quasi-static quasi-incompressible visco-elastic material behavior;
    \item Multiphase Navier-Stokes flow;
    \item The evolution of global-scale topography on planetary bodies;
    \item Goal-oriented elastoplasticity.
  \end{itemize}

\subsection{New non-standard quadrature rules}

\begin{itemize}
\item QSimplex
\item QDuffy
\item QTrianglePolar
\item QSplit
\end{itemize}

Mostly used for BEM problems.

\subsection{Incompatible changes}

\marginpar{OLD}
The 9.0 release includes around XYZ
\href{https://www.dealii.org/developer/doxygen/deal.II/changes_between_8_5_and_9_0.html}{
incompatible changes}; see \cite{changes90}. The majority of these changes
should not be visible to typical user codes; some remove previously
deprecated classes and functions, and the majority change internal
interfaces that are not usually used in external applications. However, three
incompatible changes are worth mentioning:
\begin{itemize}
  \item High-order Lagrange elements, both continuous \verb!FE_Q! and
    discontinuous \verb!FE_DGQ! types, now use the nodal points of the
    Gauss-Lobatto quadrature formula as support points by default, rather than the
    previous equidistant ones. For cubic polynomials and higher, the point
    distribution has thus changed and, consequently, the entries in
    solution vectors will be different compared to previous
    versions of \dealii{}. Note, however, that using the Gauss-Lobatto points as nodal
    points results in a much more stable interpolation, including better
    iteration counts in most iterative solvers.
  \item
    The library no longer instantiates template classes with \texttt{long
    double}. These were rarely used, but took up a significant
    fraction of compile and link time, as well as library
    size. Application programs can, however, still instantiate all
    template classes with \texttt{long
    double} as long as they include the corresponding \texttt{.templates.h}
    header files.
  \item
    The \texttt{ParameterGUI} has been moved to a separate repository.
\end{itemize}



\section{How to cite \dealii{}}\label{sec:cite}

In order to justify the work the developers of \dealii{} put into this
software, we ask that papers using the library reference one of the
\dealii{} papers. This helps us justify the effort we put into it.

There are various ways to reference \dealii{}. To acknowledge the use of the
current version of the library, \textbf{please reference the present document}. For up
to date information and bibtex snippets for this document see:
\begin{center}
 \url{https://www.dealii.org/publications.html}
\end{center}

% \begin{minipage}{0.9\textwidth}%no page break in here please
% \begin{verbatim}
% @article{dealII82,
%   title = {The {\tt deal.{I}{I}} Library, Version 8.2},
%   author = {W. Bangerth and T. Heister and L. Heltai
%    and G. Kanschat and M. Kronbichler and M. Maier
%    and B. Turcksin and T. D. Young},
%   journal = {preprint},
%   year = {2015},
% }
% \end{verbatim} %  journal = {arXiv preprint \url{http://arxiv.org/abs/TODO}},
% \end{minipage}

The original \texttt{\dealii{}} paper containing an overview of its
architecture is \cite{BangerthHartmannKanschat2007}. If you rely on specific
features of the library, please consider citing any of the following:
\begin{itemize}
 \item For geometric multigrid: \cite{Kanschat2004,JanssenKanschat2011};
 \item For distributed parallel computing: \cite{BangerthBursteddeHeisterKronbichler11};
 \item For $hp$ adaptivity: \cite{BangerthKayserHerold2007};
  \item For $PUM$ and enrichment of the FE space: \cite{Davydov2016};
 \item For matrix-free and fast assembly techniques:
   \cite{KronbichlerKormann2012};
 \item For computations on lower-dimensional manifolds:
   \cite{DeSimoneHeltaiManigrasso2009};
 \item For integration with CAD files and tools:
   \cite{HeltaiMola2015};
 \item For \texttt{LinearOperator} and \texttt{PackagedOperation} facilities:
   \cite{MaierBardelloniHeltai-2016-a,MaierBardelloniHeltai-2016-b}.
 \item For uses of the \texttt{WorkStream} interface:
   \cite{TKB16}.
\end{itemize}

\marginpar{update}
\dealii{} can interface with many other libraries:
\begin{multicols}{2}
\begin{itemize}
\item Adol-c \cite{adol-c}
\item ARPACK \cite{arpack}
\item Assimp \cite{assimp}
\item BLAS, LAPACK
\item GSL \cite{gsl2016}
\item HDF5 \cite{hdf5}
\item METIS \cite{karypis1998fast}
\item MUMPS \cite{ADE00,MUMPS:1,MUMPS:2,mumps-web-page}
\item muparser \cite{muparser-web-page}
\item nanoflann \cite{nanoflann}
\item NetCDF \cite{rew1990netcdf}
\item OpenCASCADE \cite{opencascade-web-page}
\item p4est \cite{p4est}
\item PETSc \cite{petsc-user-ref,petsc-web-page}
\item SLEPc \cite{Hernandez:2005:SSF}
\item SUNDIALS \cite{sundials}
\item Threading Building Blocks \cite{Rei07}
\item Trilinos \cite{trilinos,trilinos-web-page}
\item UMFPACK \cite{umfpack}
\end{itemize}
\end{multicols}
Please consider citing the appropriate references if you use interfaces to these
libraries.

Older releases of \dealii{} can be cited as \cite{dealII80,dealII81,dealII82,dealII83,dealII84,dealII85}.

\nocite{BangerthKanschat1999}

\section{Acknowledgments}

\dealii{} is a world-wide project with dozens of contributors around the
globe. Other than the authors of this paper, the following people contributed code to
this release:\\
%
% get this from the changes/*/* files using the command listed in the
% release-tasks paper and remove the authors of this paper
%
\marginpar{update}
  Rajat Arora,
  Mauro Bardelloni,
  Conrad Clevenger,
  Sam Cox,
  Toby D. Young,
  Juliane Dannberg,
  Nicola Demo,
  Patrick Esser,
  Niklas Fehn,
  Joscha Gedicke,
  Nicola Giuliani,
  Sebastian Gonzalez-Pintor,
  Ryan Grove,
  Michael Harmon,
  Daniel Jodlbauer,
  Guido Kanschat,
  Justin Kauffman,
  Eldar Khattatov ,
  Uwe Koecher,
  Alex Kokomov,
  Paul Kuberry,
  Dustin Kumor,
  Konstantin Ladutenko,
  Karl Ljungkvist,
  Andrew McBride,
  Mathias Mentler,
  Andrea Mola,
  Dragan Nikolic,
  Vaibhav Palkar,
  Spencer Patty,
  Jonathan Perry-Houts,
  Giuseppe Pitton,
  Ce Qin,
  Jonathan Robey,
  Mayank Sabharwal,
  Ali Samii,
  Alberto Sartori,
  Daniel Shapero,
  Martin Steigemann,
  Jihuan Tian,
  Jaeryun Yim,
  Liang Zhao


Their contributions are much appreciated!


\bigskip

\dealii{} and its developers are financially supported through a
variety of funding sources:
\marginpar{update}

D.~Arndt was partially supported by the German Research Foundation (DFG) under the
project ``High-order discontinuous Galerkin for the exa-scale'' (ExaDG) within the
priority program ``Software for Exascale Computing'' (SPPEXA).

% W.~Bangerth was partially
% supported by the National Science Foundation under award OCI-1148116
% as part of the Software Infrastructure for Sustained Innovation (SI2)
% program; and by the Computational Infrastructure in Geodynamics initiative
% (CIG), through the National Science Foundation under Awards
% No.~EAR-0949446 and EAR-1550901 and The University of California -- Davis.
%
V.~Boddu was supported by the German Science Foundation (Deutsche 
Forschungs-Gemeinschaft, DFG) under the research group project FOR 1509.

B.~Brands was partially supported by the Indo-German exchange programm ``Multiscale Modeling, Simulation and Optimization for Energy, Advanced Materials and Manufacturing'' (MMSO) funded by DAAD (Germany) and UGC (India).

D.~Davydov was supported by the German Science Foundation
(Deutsche Forschungs-Gemeinschaft, DFG), grant DA 1664/2-1.

T.~Heister was partially supported by the Computational Infrastructure in
Geodynamics initiative (CIG), through the National Science Foundation
under Award No. EAR-0949446 and EAR-1550901 and The University of California -- Davis, and National Science Foundation grant DMS-1522191.

M.~Kronbichler was partially supported by the German Research Foundation (DFG)
under the project ``High-order discontinuous Galerkin for the exa-scale''
(ExaDG) within the priority program ``Software for Exascale Computing''
(SPPEXA), grant agreement no.~KR4661/2-1 and the Bayerisches Kompetenznetzwerk
f\"ur Technisch-Wissenschaftliches Hoch- und H\"ochstleistungsrechnen
(KONWIHR).

% J-P.~Pelteret was supported by the European Research Council (ERC) through the Advanced Grant 289049 MOCOPOLY.
%
B.~Turcksin: This material is based upon work supported by the U.S. Department of
Energy, Office of Science, under contract number DE-AC05-00OR22725.

D.~Wells was supported by the National Science Foundation (NSF) through Grant
DMS-1344962.

The Interdisciplinary Center for Scientific Computing (IWR) at Heidelberg University has provided
hosting services for the \dealii{} web page.


\bibliography{paper}{}
\bibliographystyle{abbrv}

\end{document}